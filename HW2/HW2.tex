\documentclass[11pt]{article}
\usepackage{../EllioStyle}
\usepackage{listings}

\definecolor{codegreen}{rgb}{0,0.6,0}
\definecolor{codegray}{rgb}{0.5,0.5,0.5}
\definecolor{codepurple}{rgb}{0.58,0,0.82}
\definecolor{backcolour}{rgb}{0.95,0.95,0.92}

\lstdefinestyle{mystyle}{
%    backgroundcolor=\color{backcolour},   
    commentstyle=\color{codegreen},
    keywordstyle=\color{magenta},
    numberstyle=\tiny\color{codegray},
    stringstyle=\color{codepurple},
    basicstyle=\ttfamily\footnotesize,
    breakatwhitespace=false,         
    breaklines=true,                 
    captionpos=b,                    
    keepspaces=true,                 
    numbers=left,                    
    numbersep=5pt,                  
    showspaces=false,                
    showstringspaces=false,
    showtabs=false,                  
    tabsize=2
}

\title{Homework 2}
\author{Elliott Pryor}
\date{7 September 2021}

\rhead{Homework 2}

\begin{document}
\maketitle

\problem{1}

Propose a dynamic programming algorithm to solve longest common substring problem
\hrule



\problem{2}
Given two sequences S, T (not necessarily the same length), let G, L, H be the scores of the
optimal global alignment, optimal local alignment, and optimal global alignment without penalizing leading or trailing spaces.


\begin{enumerate}[a)]
    \item Give an example of S, T so that the three scores are different
    \item Prove or disprove the statement $L \geq H \geq G$
\end{enumerate}


\problem{3}

Implement the global alignment algorithm from class.  Your program should read in a FASTA file (see HW1).  You can assume that the file just contains two sequences, e.g.

> seq1
ACTGGGAAA
> seq2
CTGGAACA

The filename should be supplied as a command-line parameter.

Align the first string with the second string.  Print out one optimal alignment.

You can assume a simplified scoring function delta that has the following form:
delta(match) = 2
delta(mismatch) = -1
delta(insertion/deletion) = -1

Demonstrate your algorithms on two test cases (use screen shots to show runs). 

\hrule



\lstset{style=mystyle}
\lstinputlisting[language=Python]{EditDistance.py}


\end{document}